\documentclass[paper=a4, fontsize=11pt]{scrartcl} % A4 paper and 11pt font size
\usepackage{./../usfassignment}
\settitle{Assignment 5}
\setauthor{Wanzhang Sheng}
\setcourse{CS315: Computer Architecture}

\begin{document}

\maketitle % Print the title

% -----------------------------------------------------------------------------
% PROBLEM  1
% -----------------------------------------------------------------------------
\section{2.24}

\begin{fancyquotes}

  Suppose the program counter (PC) is set to $\mathtt{0x2000 0000}$. Is it
  possible to use the jump (j) MIPS assembly instruction to set the PC to the
  address as $\mathtt{0x4000 0000}$? Is it possible to use the branch-on-equal
  (beq) MIPS assembly instruction to set the PC to this same address?

\end{fancyquotes}

$\mathtt{0x4000 0000} - \mathtt{0x2000 0000} > 2^{28}$

Though jump can locate to full 32bits address, it's not possible to jump to a
address greater than 28bits with one single jump instruction.

It's not possible to branch, since branch is I-type instruction, which can only
use 18bits address.


% -----------------------------------------------------------------------------
% PROBLEM  2
% -----------------------------------------------------------------------------
\section{2.40}

\begin{fancyquotes}

  If the current value of the PC is $\mathtt{0x0000\ 0000}$, can you use a
  single jump instruction to get to the PC address $(\mathtt{0010\ 0000\ 0000\
  0001\ 0100\ 1001\ 0010\ 0100})_2$.

\end{fancyquotes}

$((\mathtt{0010\ 0000\ 0000\ 0001\ 0100\ 1001\ 0010\ 0100})_2
- \mathtt{0x0000\ 0000})
= \mathtt{0x20014924}
= \mathtt{536955172}
> 2^{28}$

It's not possible to jump with a single jump instruction.


% -----------------------------------------------------------------------------
% PROBLEM  3
% -----------------------------------------------------------------------------
\section{2.41}

\begin{fancyquotes}

  If the current value of the PC is $\mathtt{0x00000600}$, can you use a single
  branch instruction to get to the PC address $(\mathtt{0010\ 0000\ 0000\ 0001\
  0100\ 1001\ 0010\ 0100})_2$9?

\end{fancyquotes}

$((\mathtt{0010\ 0000\ 0000\ 0001\ 0100\ 1001\ 0010\ 0100})_2
- \mathtt{0x0000\ 0600})
= \mathtt{0x20014324}
= \mathtt{536953636}
> 2^{18}$

It's not possible to branch with a single branch instruction.


% -----------------------------------------------------------------------------
% PROBLEM  4
% -----------------------------------------------------------------------------
\section{2.42}

\begin{fancyquotes}

  If the current value of the PC is $\mathtt{0x1FFF\ F000}$, can you use a
  single branch instruction to get to the PC address $(\mathtt{0010\ 0000\ 0000\
  0001\ 0100\ 1001\ 0010\ 0100})_2$?

\end{fancyquotes}

$((\mathtt{0010\ 0000\ 0000\ 0001\ 0100\ 1001\ 0010\ 0100})_2
- \mathtt{0x1FFF\ F000})
= \mathtt{0x0001\ 5924}
= \mathtt{88356}
< 2^{18}$

So it's possible to branch with a single branch instruction.


\end{document}
