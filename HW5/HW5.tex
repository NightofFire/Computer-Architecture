\documentclass[paper=a4, fontsize=11pt]{scrartcl} % A4 paper and 11pt font size
\usepackage{./../usfassignment}
\usepackage{multirow}
\settitle{Assignment 5}
\setauthor{Wanzhang Sheng}
\setcourse{CS315: Computer Architecture}

\begin{document}

\maketitle % Print the title

% -----------------------------------------------------------------------------
% PROBLEM  1
% -----------------------------------------------------------------------------
\section{2.24}

\begin{fancyquotes}

  Suppose the program counter (PC) is set to $\mathtt{0x2000 0000}$. Is it
  possible to use the jump (j) MIPS assembly instruction to set the PC to the
  address as $\mathtt{0x4000 0000}$? Is it possible to use the branch-on-equal
  (beq) MIPS assembly instruction to set the PC to this same address?

\end{fancyquotes}

$\mathtt{0x4000 0000} - \mathtt{0x2000 0000} > 2^{28}$

Though jump can locate to full 32bits address, it's not possible to jump to a
address greater than 28bits with one single jump instruction.

It's not possible to branch, since branch is I-type instruction, which can only
use 18bits address.


% -----------------------------------------------------------------------------
% PROBLEM  2
% -----------------------------------------------------------------------------
\section{2.40}

\begin{fancyquotes}

  If the current value of the PC is $\mathtt{0x0000\ 0000}$, can you use a
  single jump instruction to get to the PC address $(\mathtt{0010\ 0000\ 0000\
  0001\ 0100\ 1001\ 0010\ 0100})_2$.

\end{fancyquotes}

$((\mathtt{0010\ 0000\ 0000\ 0001\ 0100\ 1001\ 0010\ 0100})_2
- \mathtt{0x0000\ 0000})
= \mathtt{0x20014924}
= \mathtt{536955172}
> 2^{28}$

It's not possible to jump with a single jump instruction.


% -----------------------------------------------------------------------------
% PROBLEM  3
% -----------------------------------------------------------------------------
\section{2.41}

\begin{fancyquotes}

  If the current value of the PC is $\mathtt{0x00000600}$, can you use a single
  branch instruction to get to the PC address $(\mathtt{0010\ 0000\ 0000\ 0001\
  0100\ 1001\ 0010\ 0100})_2$9?

\end{fancyquotes}

$((\mathtt{0010\ 0000\ 0000\ 0001\ 0100\ 1001\ 0010\ 0100})_2
- \mathtt{0x0000\ 0600})
= \mathtt{0x20014324}
= \mathtt{536953636}
> 2^{18}$

It's not possible to branch with a single branch instruction.


% -----------------------------------------------------------------------------
% PROBLEM  4
% -----------------------------------------------------------------------------
\section{2.42}

\begin{fancyquotes}

  If the current value of the PC is $\mathtt{0x1FFF\ F000}$, can you use a
  single branch instruction to get to the PC address $(\mathtt{0010\ 0000\ 0000\
  0001\ 0100\ 1001\ 0010\ 0100})_2$?

\end{fancyquotes}

$((\mathtt{0010\ 0000\ 0000\ 0001\ 0100\ 1001\ 0010\ 0100})_2
- \mathtt{0x1FFF\ F000})
= \mathtt{0x0001\ 5924}
= \mathtt{88356}
< 2^{18}$

So it's possible to branch with a single branch instruction.


% -----------------------------------------------------------------------------
% PROBLEM  5
% -----------------------------------------------------------------------------
\section{}

\begin{fancyquotes}

  Procedure A has a text size of $\mathtt{0x140}$ and a data size of of
  $\mathtt{0x40}$. Procedure B has a text size of $\mathtt{0x300}$ and a data
  size of $\mathtt{0x50}$. Show the executables created by the linker for these
  two pairs of procedures if the linker uses the memory layout shown in the MIPS
  Green sheet.

\end{fancyquotes}

\subsection{a}

\begin{table}[h]
  \begin{center}
    \begin{tabular}{|c|c|c|}
    \hline
    \textbf{Executable} & & \\
    \hline
    \multirow{2}{*}{Header} & Text size & $\mathtt{0x440}$ \\
    \cline{2-3}
                            & Data size & $\mathtt{0x90}$ \\
    \hline
    \multirow{2}{*}{Text Segment} & Address & Instruction \\
    \cline{2-3}
                            & $\mathtt{0x0040\ 0000}$ & lbu \$a0, $\mathtt{0x8000}$(\$gp) \\
    \cline{2-3}
                            & $\mathtt{0x0040\ 0004}$ & jal $\mathtt{0x10\ 0050}$ \\
    \cline{2-3}
                            & $\mathtt{0x0040\ 0140}$ & sw \$a1, $\mathtt{0x8040}$(\$gp) \\
    \cline{2-3}
                            & $\mathtt{0x0040\ 0144}$ & jal $\mathtt{0x10\ 0000}$ \\
    \hline
    \multirow{2}{*}{Data Segment} & Address & \\
    \cline{2-3}
                            & $\mathtt{0x1000\ 0000}$ & (X) \\
    \cline{2-3}
                            & $\mathtt{0x1000\ 0040}$ & (Y) \\
    \hline
    \end{tabular}
  \end{center}
\end{table}

\subsection{b}

\begin{table}[h]
  \begin{center}
    \begin{tabular}{|c|c|c|}
    \hline
    \textbf{Executable} & & \\
    \hline
    \multirow{2}{*}{Header} & Text size & $\mathtt{0x440}$ \\
    \cline{2-3}
                            & Data size & $\mathtt{0x90}$ \\
    \hline
    \multirow{2}{*}{Text Segment} & Address & Instruction \\
    \cline{2-3}
                            & $\mathtt{0x0040\ 0000}$ & lui \$at, $\mathtt{0x1000}$ \\
    \cline{2-3}
                            & $\mathtt{0x0040\ 0004}$ & ori \$a0, \$at, $\mathtt{0x0000}$ \\
    \cline{2-3}
                            & $\cdots$ & $\cdots$ \\
    \cline{2-3}
                            & $\mathtt{0x0040\ 0084}$ & jr \$ra \\
    \cline{2-3}
                            & $\cdots$ & $\cdots$ \\
    \cline{2-3}
                            & $\mathtt{0x0040\ 0140}$ & sw \$a0, $\mathtt{0x8040}$(\$gp) \\
    \cline{2-3}
                            & $\mathtt{0x0040\ 0144}$ & jmp $\mathtt{0x0010\ 00b0}$ \\
    \cline{2-3}
                            & $\cdots$ & $\cdots$ \\
    \cline{2-3}
                            & $\mathtt{0x0040\ 02c0}$ & jal $\mathtt{0x0010\ 0000}$ \\
    \cline{2-3}
                            & $\cdots$ & $\cdots$ \\
    \hline
    \multirow{2}{*}{Data Segment} & Address & \\
    \cline{2-3}
                            & $\mathtt{0x1000\ 0000}$ & (X) \\
    \cline{2-3}
                            & $\mathtt{0x1000\ 0040}$ & (Y) \\
    \hline
    \end{tabular}
  \end{center}
\end{table}

\end{document}
