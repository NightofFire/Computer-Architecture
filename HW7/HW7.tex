\documentclass[paper=a4, fontsize=11pt]{scrartcl} % A4 paper and 11pt font size
\usepackage{./../usfassignment}
\usepackage{./../mips}
\settitle{Assignment 7}
\setauthor{Wanzhang Sheng}
\setcourse{CS315: Computer Architecture}

\begin{document}

\maketitle % Print the title

MIPS version binary semaphores:

\lstinputlisting[language={[mips]Assembler}]{pth_msg_sem_mips.s}

% -----------------------------------------------------------------------------
% PROBLEM  1
% -----------------------------------------------------------------------------
\section{}

\begin{fancyquotes}
    Subtraction of two unsigned decimal integers:
\end{fancyquotes}

\begin{lstlisting}[language=C]
    update = minu;
    for (digit = 0; digit < max_digits; digit++) {
       if (update[digit] < subt[digit]) {
          update[digit] += 10;
          i = digit + 1;
          while ((i < max_digits) && (update[i] == 0)) {
             update[i] = 9;
             i++;
          }
          if (i >= max_digits) exit(127); // Overflow
          update[i]--;
       }
       diff[digit] = update[digit] - subt[digit];
    }
\end{lstlisting}

If we use a marker to indicate the result is negative:

\begin{lstlisting}[language=C]
    update = minu;
    for (digit = 0; digit < max_digits; digit++) {
       if (update[digit] < subt[digit]) {
          update[digit] += 10;
          i = digit + 1;
          while ((i < max_digits) && (update[i] == 0)) {
             update[i] = 9;
             i++;
          }
          if (i >= max_digits) {
            // Fix the digits to negative number
            int j;
            for (j = 0; i < max_digits-1; j++)
              update[j] = 9 - update[j];
            // set the highest digit to -1 to indicate it is negative
            update[max_digits-1] = -1;
            return update;
          }
          update[i]--;
       }
       diff[digit] = update[digit] - subt[digit];
    }
\end{lstlisting}


% -----------------------------------------------------------------------------
% PROBLEM  2
% -----------------------------------------------------------------------------
\section{3.1}

\begin{fancyquotes}
  What is $\texttt{5ED4} - \texttt{07A4}$ when these values represent unsigned 16-bit hexadecimal numbers? The result should be written in hexadecimal. Show your work.
\end{fancyquotes}

$\texttt{0x5ED4} - \texttt{0x07A4} = 24276 - 1956 = 22320 = \texttt{5730}$


% -----------------------------------------------------------------------------
% PROBLEM  3
% -----------------------------------------------------------------------------
\section{3.2}

\begin{fancyquotes}
  What is $\texttt{5ED4} - \texttt{07A4}$ when these values represent signed 16-bit hexadecimal numbers stored in sign-magnitude format? The result should be written in hexadecimal. Show your work.
\end{fancyquotes}

$\texttt{0x5ED4} - \texttt{0x07A4} = 24276 - 1956 = 22320 = \texttt{5730}$

Since $\texttt{0x5} = \texttt{0101}$, first bit is $0$, so these three numbers are all positive.

% -----------------------------------------------------------------------------
% PROBLEM  4
% -----------------------------------------------------------------------------
\section{3.6}

\begin{fancyquotes}
  Assume $185$ and $122$ are unsigned 8-bit decimal integers. Calculate $185-122$. Is there overflow, underflow, or neither?
\end{fancyquotes}

$185 - 122 = 63$

Since unsigned 8-bit integers range is $0~255$, they are all in range.


% -----------------------------------------------------------------------------
% PROBLEM  5
% -----------------------------------------------------------------------------
\section{3.7}

\begin{fancyquotes}
  Assume $185$ and $122$ are signed 8-bit decimal integers stored in sign-magnitude format. Calculate $185+122$. Is there overflow, underflow, or neither?
\end{fancyquotes}

Signed 8-bit integers range is $-128~127$, so $185$ is actually $-71$.

$-71 + 122 = 51$

Which the result is in the range.


% -----------------------------------------------------------------------------
% PROBLEM  6
% -----------------------------------------------------------------------------
\section{3.8}

\begin{fancyquotes}
  Assume $185$ and $122$ are signed 8-bit decimal integers stored in sign-magnitude format. Calculate $185-122$. Is there overflow, underflow, or neither?
\end{fancyquotes}

Since $185$ is actually $-71$ in signed 8-bit integers.

$-71 - 122 = -193$

Which the result is underflow.


\end{document}
