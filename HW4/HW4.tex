\documentclass[paper=a4, fontsize=11pt]{scrartcl} % A4 paper and 11pt font size
\usepackage{../usfassignment}
\usepackage{../mips}
\settitle{Assignment 4}
\setauthor{Wanzhang Sheng}
\setcourse{CS315: Computer Architecture}

\begin{document}

\maketitle % Print the title

% -----------------------------------------------------------------------------
% PROBLEM  1
% -----------------------------------------------------------------------------
\section{2.8}

\begin{fancyquotes}
  Translate $\mathtt{0xabcdef12}$ into decimal.
\end{fancyquotes}

$$\mathtt{0xabcdef12} =
10\times 16^7 +
11\times 16^6 +
12\times 16^5 +
13\times 16^4 +
14\times 16^3 +
15\times 16^2 +
1\times 16^1 +
2\times 16^0
= {(2882400018)}_{10}$$


% -----------------------------------------------------------------------------
% PROBLEM 2
% -----------------------------------------------------------------------------
\section{2.11}

\begin{fancyquotes}
  For each MIPS instruction, show the value of the opcode (OP), source
  register (RS), and target register (RT) fields. For the I-type
  instructions, show the value of the immediate field, and for the
  R-type instructions, show the value of the destination register (RD)
  field.
\end{fancyquotes}

\begin{lstlisting}[language={[mips]Assembler}]
  addi    $t0, $s6, 4	# I-Type: OP: 8, RS: 22, RT: 8, IM: 4
  add     $t1, $s6, $0	# R-Type: OP: 32, RS: 0, RT: 22, RD: 9
  sw      $t1, 0($t0)	# I-Type: OP: 43, RS: 8, RT: 9, IM: 0
  lw      $t0, 0($t0)	# I-Type: OP: 35, RS: 8, RT: 8, IM: 0
  add     $s0, $t1, $t0	# R-Type: OP: 32, RS: 8, RT: 9, RD: 16
\end{lstlisting}


% -----------------------------------------------------------------------------
% PROBLEM 3
% -----------------------------------------------------------------------------
\section{2.12.3}

\begin{fancyquotes}
  Assume that registers \$s0 and \$s1 hold the values
  $\mathrm{0x80000000}$ and $\mathrm{0xD0000000}$, respectively.

  For the contents of registers \$s0 and \$s1 as specified above, what
  is the value of \$t0 for the following assembly code?
\end{fancyquotes}

\begin{lstlisting}[language={[mips]Assembler}]
  sub $t0, $s0, $s1
\end{lstlisting}                %$

$\$t0 = \$s0 - \$s1 = \mathrm{0x80000000} - \mathrm{0xD0000000}
= 2147483648 - 3489660928
= -1342177280 = \mathrm{0xA0000000}$

% -----------------------------------------------------------------------------
% PROBLEM 4
% -----------------------------------------------------------------------------
\section{2.14}

\begin{fancyquotes}
  Provide the type and assembly language instruction for the following
  binary value: ${(0000\ 0010\ 0001\ 0000\ 1000\ 0000\ 0010\ 0000)}_2$
\end{fancyquotes}

opcode: $\mathrm{0x000000} = 0$

operation: add

type: $R$

RS: ${(10000)}_2 = 16$, \$s0

RT: ${(10000)}_2 = 16$, \$s0

RD: ${(10000)}_2 = 16$, \$s0

shamt: ${(000000)}_2 = 0$

funct: ${(100000)}_2 = 32$

\begin{lstlisting}[language={[mips]Assembler}]
  add	$s0, $s0, $s0
\end{lstlisting}                %$


% -----------------------------------------------------------------------------
% PROBLEM 5
% -----------------------------------------------------------------------------
\section{2.15}

\begin{fancyquotes}
  Provide the type and hexadecimal representation of following instruction:
\end{fancyquotes}

\begin{lstlisting}[language={[mips]Assembler}]
  sw	$t1, 32($t2)
\end{lstlisting}


I-Type, machine code: ${(101011\ 01010\ 01001\ 00000\ 00000\ 100000)}_2$,
hex as $\mathrm{0xAD490020}$.


% -----------------------------------------------------------------------------
% PROBLEM 6
% -----------------------------------------------------------------------------
\section{2.16}

\begin{fancyquotes}
  Provide the type, assembly language instruction, and binary
  representation of instruction described by the following MIPS
  fields:

  op=0, rs=3, rt=2, rd=3, shamt=0, funct=34
\end{fancyquotes}

Binary: ${(000000\ 00011\ 00010\ 00011\ 00000\ 000022)}_2$

Type: R

Assembly:
\begin{lstlisting}[language={[mips]Assembler}]
  sub	$v1, $v0, $v1
\end{lstlisting}                %$


% -----------------------------------------------------------------------------
% PROBLEM 7
% -----------------------------------------------------------------------------
\section{2.17}

\begin{fancyquotes}
  Provide the type, assembly language instruction, and binary
  representation of instruction described by the following MIPS
  fields:
\end{fancyquotes}

op=$\mathrm{0x23}$, rs=$1$, rt=$2$, const=$\mathrm{0x4}$

Type: I-Type

Assembly:
\begin{lstlisting}[language={[mips]Assembler}]
  lw	$v0, 4($at)
\end{lstlisting}                %$

Binary: ${(100011\ 00001\ 00010\ 00000\ 00000\ 000100)}_2$


% -----------------------------------------------------------------------------
% PROBLEM 8
% -----------------------------------------------------------------------------
\section{2.21}

\begin{fancyquotes}
  Provide a minimal set of MIPS instructions that may be used to
  implement the following pseudoinstruction:
\end{fancyquotes}

\begin{lstlisting}[language={[mips]Assembler}]
  not	$t1, $t2 # bit-wise invert
\end{lstlisting}                %$

can be implemented as:

\begin{lstlisting}[language={[mips]Assembler}]
  nor	$t1, $t2, $zero
\end{lstlisting}                %$


% -----------------------------------------------------------------------------
% PROBLEM 9
% -----------------------------------------------------------------------------
\section{converter}

\subsection{}
\begin{fancyquotes}
  Convert $1234$ from decimal into binary
\end{fancyquotes}

${(1234)}_{10} = {(10011010010)}_2$

\subsection{}
\begin{fancyquotes}
  Convert $4321$ from decimal into hexadecimal
\end{fancyquotes}

${(4321)}_{10} = {(1000011100001)}_2$

\subsection{}
\begin{fancyquotes}
  Convert ${(0110\ 1011)}_2$ from binary into decimal
\end{fancyquotes}

${(0110\ 1011)}_2 = {(107)}_{10}$

\subsection{}
\begin{fancyquotes}
  Convert $\mathrm{0x1234}$ from hexadecimal into decimal
\end{fancyquotes}

$\mathrm{0x1234} = {(4660)}_{10}$

\subsection{}
\begin{fancyquotes}
  In an 8-bit system find the two's complement representation of the
  decimal value $-21$.
\end{fancyquotes}

$${(21)}_{10}  = {(0001\ 0101)}_2$$
$${(-21)}_{10} = {(1110\ 1011)}_2$$

\subsection{}
\begin{fancyquotes}
  In a 16-bit system find the two's complement representation of the
  decimal value $-21$.
\end{fancyquotes}

$${(21)}_{10}  = {(0000\ 0000\ 0001\ 0101)}_2$$
$${(-21)}_{10} = {(1111\ 1111\ 1110\ 1011)}_2$$

\end{document}
