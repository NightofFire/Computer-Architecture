\documentclass[paper=a4, fontsize=11pt]{scrartcl} % A4 paper and 11pt font size
\usepackage{./../usfassignment}
\usepackage{./../mips}
\usepackage{multirow}
\settitle{Assignment 8}
\setauthor{Wanzhang Sheng}
\setcourse{CS315: Computer Architecture}

\begin{document}

\maketitle % Print the title

% -----------------------------------------------------------------------------
% PROBLEM  1
% -----------------------------------------------------------------------------
\section{Test}

\begin{fancyquotes}
    Run the program and indicate in the program documentation how well the proposed algorithm compares with the hardware multiply. How many iterations of the two methods did you use?
\end{fancyquotes}

\begin{lstlisting}[language=bash]
    gcc -Wall -o compare_mult compare_mult.c && ./compare_mult 1000000
    Time for hardware = 7.414818e-03 seconds
    Time for proposed = 1.511827e+01 seconds
\end{lstlisting}

Let $m$ be the number of bits of multiplier, and $n$ be the number of bits of multiplicand.

Then the hardware version runs $m\times n$ times iterations.
And the proposed version runs $2^m\times n$ times iterations.

\pagebreak


% -----------------------------------------------------------------------------
% PROBLEM  2
% -----------------------------------------------------------------------------
\section{3.12}

\begin{fancyquotes}
    Using a table similar to that shown in Figure 3.7, calculate the product of the octal unsigned 6-bit integers 62 and 12 using the hardware described in Figure 3.4. You should show the contents of each register on each step.
\end{fancyquotes}

${(62)}_8 \times {(12)}_8 = {(764)}_8$

\begin{tabular}{clccc}
  Iteration           & Step            & Multiplier & Multiplicand & Product \\
  \toprule
  0                   & Initial values     & 110010  & 0000 0000 1010  & 0000 0000 0000 \\
  \midrule
  \multirow{3}{*}{1}  & 0: Do nothing      & 110010  & 0000 0000 1010  & 0000 0000 0000 \\
                      & Shift Multiplier   & 011001  & 0000 0000 1010  & 0000 0000 0000 \\
                      & Shift Multiplicand & 011001  & 0000 0001 0100  & 0000 0000 0000 \\
  \midrule
  \multirow{3}{*}{2}  & 1: Plus            & 011001  & 0000 0001 0100  & 0000 0001 0100 \\
                      & Shift Multiplier   & 001100  & 0000 0001 0100  & 0000 0001 0100 \\
                      & Shift Multiplicand & 001100  & 0000 0010 1000  & 0000 0001 0100 \\
  \midrule
  \multirow{3}{*}{3}  & 0: Do nothing      & 001100  & 0000 0010 1000  & 0000 0001 0100 \\
                      & Shift Multiplier   & 000110  & 0000 0010 1000  & 0000 0001 0100 \\
                      & Shift Multiplicand & 000110  & 0000 0101 0000  & 0000 0001 0100 \\
  \midrule
  \multirow{3}{*}{4}  & 0: Do nothing      & 000110  & 0000 0101 0000  & 0000 0001 0100 \\
                      & Shift Multiplier   & 000011  & 0000 0101 0000  & 0000 0001 0100 \\
                      & Shift Multiplicand & 000011  & 0000 1010 0000  & 0000 0001 0100 \\
  \midrule
  \multirow{3}{*}{5}  & 1: Plus            & 000011  & 0000 1010 0000  & 0000 1011 0100 \\
                      & Shift Multiplier   & 000001  & 0000 1010 0000  & 0000 1011 0100 \\
                      & Shift Multiplicand & 000001  & 0001 0100 0000  & 0000 1011 0100 \\
  \midrule
  \multirow{3}{*}{6}  & 1: Plus            & 000001  & 0001 0100 0000  & 0001 1111 0100 \\
                      & Shift Multiplier   & 000000  & 0010 1000 0000  & 0001 1111 0100 \\
                      & Shift Multiplicand & 000000  & 0010 1000 0000  & 0001 1111 0100 \\
  \bottomrule
\end{tabular}

\pagebreak

% -----------------------------------------------------------------------------
% PROBLEM  3
% -----------------------------------------------------------------------------
\section{Code}

\begin{fancyquotes}
    compare\_mult.c
\end{fancyquotes}

\begin{lstlisting}[language=C]
  int Proposed(int x, int y) {
     // multiplier and multiplicand >= 0
     int product = 0;
     for (int i = 0; i < x; i++)
        product += y;
     return product;
  }  /* Proposed */
\end{lstlisting}

\begin{fancyquotes}
    ovflow\_det.c
\end{fancyquotes}

\begin{lstlisting}[language=C]
  long Overflow(long m, long n) {
     return (m > 0 && n > 0 && (m + n) < 0) || (m < 0 && n < 0 && (m + n) > 0);
  }
\end{lstlisting}

\begin{fancyquotes}
    ovflow\_det.s
\end{fancyquotes}

\begin{lstlisting}[language={[mips]Assembler}]
  # Function:     Overflow
  # Purpose:      Determine whether the sum of the two arguments has
  #               overflowed.
  #
  # C Prototype:  long Overflow(long m, long n);
  # Input args:   rdi = m, rsi = n
  # Ret val:      0 if the sum doesn't overflow, nonzero otherwise
  #
          .section .text
          .global Overflow
  Overflow:
          push    %rbp
          mov     %rsp, %rbp

          addq    %rsi, %rdi      # rdi = m + n
          jo      over
          mov     $0, %rax        # return 0
          jmp     done
  over:
          mov     $1, %rax        # return 1
  done:
          leave
          ret
\end{lstlisting}

\pagebreak

\end{document}
